%%%%%%%%%%%%%%%%%%%%%%%%%%%%%%%%%%%%%%%%%
% Short Sectioned Assignment
% LaTeX Template
% Version 1.0 (5/5/12)
%
% This template has been downloaded from:
% http://www.LaTeXTemplates.com
%
% Original author:
% Frits Wenneker (http://www.howtotex.com)
%
% License:
% CC BY-NC-SA 3.0 (http://creativecommons.org/licenses/by-nc-sa/3.0/)
%
%%%%%%%%%%%%%%%%%%%%%%%%%%%%%%%%%%%%%%%%%

%----------------------------------------------------------------------------------------
%	PACKAGES AND OTHER DOCUMENT CONFIGURATIONS
%----------------------------------------------------------------------------------------

\documentclass[paper=a4, fontsize=11pt]{scrartcl} % A4 paper and 11pt font size

\usepackage[T1]{fontenc} % Use 8-bit encoding that has 256 glyphs
\usepackage{fourier} % Use the Adobe Utopia font for the document - comment this line to return to the LaTeX default
\usepackage[english]{babel} % English language/hyphenation
\usepackage{amsmath,amsfonts,amsthm} % Math packages

\usepackage{lipsum} % Used for inserting dummy 'Lorem ipsum' text into the template

\usepackage{sectsty} % Allows customizing section commands
\allsectionsfont{\normalfont\scshape} % Make all sections centered, the default font and small caps

\usepackage{fancyhdr} % Custom headers and footers
\pagestyle{fancyplain} % Makes all pages in the document conform to the custom headers and footers
\fancyhead{} % No page header - if you want one, create it in the same way as the footers below
\fancyfoot[L]{} % Empty left footer
\fancyfoot[C]{} % Empty center footer
\fancyfoot[R]{\thepage} % Page numbering for right footer
\renewcommand{\headrulewidth}{0pt} % Remove header underlines
\renewcommand{\footrulewidth}{0pt} % Remove footer underlines
\setlength{\headheight}{13.6pt} % Customize the height of the header

% \numberwithin{equation}{section} % Number equations within sections (i.e. 1.1, 1.2, 2.1, 2.2 instead of 1, 2, 3, 4)
% \numberwithin{figure}{section} % Number figures within sections (i.e. 1.1, 1.2, 2.1, 2.2 instead of 1, 2, 3, 4)
% \numberwithin{table}{section} % Number tables within sections (i.e. 1.1, 1.2, 2.1, 2.2 instead of 1, 2, 3, 4)

\setlength\parindent{0pt} % Removes all indentation from paragraphs - comment this line for an assignment with lots of text

% Custom packages
\usepackage{xeCJK} 	% Chinese characters support
\setCJKmainfont{Hiragino Sans GB}

\usepackage{float}
\usepackage{hyperref}

\usepackage{listings}
\lstset{ 
    language=Matlab,                        % choose the language of the code
%   basicstyle=10pt,                    % the size of the fonts that are used for the code
    numbers=left,                           % where to put the line-numbers
    numberstyle=\footnotesize,              % the size of the fonts that are used for the line-numbers
    stepnumber=1,                               % the step between two line-numbers. If it's 1 each line will be numbered
    numbersep=5pt,                          % how far the line-numbers are from the code
%   backgroundcolor=\color{white},      % choose the background color. You must add \usepackage{color}
    showspaces=false,                       % show spaces adding particular underscores
    showstringspaces=false,                 % underline spaces within strings
    showtabs=false,                             % show tabs within strings adding particular underscores
%   frame=single,                               % adds a frame around the code
%   tabsize=2,                              % sets default tabsize to 2 spaces
%   captionpos=b,                               % sets the caption-position to bottom
    breaklines=true,                            % sets automatic line breaking
    breakatwhitespace=false,                % sets if automatic breaks should only happen at whitespace
    escapeinside={\%*}{*)}                  % if you want to add a comment within your code
}


%----------------------------------------------------------------------------------------
%	TITLE SECTION
%----------------------------------------------------------------------------------------

\newcommand{\horrule}[1]{\rule{\linewidth}{#1}} % Create horizontal rule command with 1 argument of height

\title{	
\normalfont \normalsize 
\textsc{VLSI Digital Signal Processing} \\ [25pt] % Course name
\horrule{0.5pt} \\[0.4cm] % Thin top horizontal rule
\huge 第2章 迭代边界 \\ % The assignment title
\horrule{2pt} \\[0.5cm] % Thick bottom horizontal rule
}

\author{郭一隆} % Your name

\date{\normalsize\today} % Today's date or a custom date

\begin{document}

\maketitle % Print the title

\section{整体思路}
\subsection{LPM算法实现} % (fold)
\label{sub:lpm算法实现}

利用\texttt{MATLAB}实现LPM算法,以\texttt{d}(延迟节点数目)和$L^{(1)}$作为输入,详见附录\texttt{LPM.m}。

\textbf{注:将LPM算法中~-1的定义进一步修改为~$-\infty$,方便MATLAB运算。}

% subsection lpm算法实现 (end)

\subsection{$L^{(1)}$求解} % (fold)
\label{sub:L1求解}

\begin{itemize}
    \item 对于第1题、第3题,DFG结构较简单,$L^{(1)}$可直接得到;
    \item 对于第5题,DFG结构复杂,用递归的动态规划算法求解任意两个延迟节点之间的最长路径(不经过其他延迟节点),从而获得$L^{(1)}$,详见附录\texttt{p5\_graph.m}。
\end{itemize}

% subsection L1求解 (end)

\subsection{$L$序列求解} % (fold)
\label{sub:}

利用$L^{(m)}$与$L^{(\left\lfloor m/2\right\rfloor)}$、$L^{(\left\lceil m/2\right\rceil)}$的关系进行递归求解,降低复杂度。

% subsection  (end)

\section{问题求解} % (fold)
\label{sec:问题求解}

\subsection{Problem 1} % (fold)
\label{sub:1}

输入:
\begin{itemize}
    \item $d=3$
    \item $L^{(1)} = \begin{bmatrix}
        -\infty & 0 & -\infty \\
        7 & -\infty & 3 \\
        3 & -\infty & -\infty \\
    \end{bmatrix} $
\end{itemize}

运行\texttt{iteration\_bound.m}可得输出:
\begin{itemize}
    \item \texttt{iteration\_bound = 3.5}
\end{itemize}

% subsection 1 (end)

\subsection{Problem 3} % (fold)
\label{sub:problem_3}

输入:
\begin{itemize}
    \item $d=6$
    \item $L^{(1)} = \begin{bmatrix}
        4 & 4 & 4 & -\infty & 4 & -\infty \\
        -\infty & -\infty & -\infty & -\infty & -\infty & -\infty \\
        -\infty & -\infty & -\infty & 0 & -\infty & -\infty \\
        4 & 4 & 4 & -\infty & 4 & -\infty \\
        -\infty & -\infty & -\infty & -\infty & -\infty & 0 \\
        -\infty & -\infty & -\infty & -\infty & -\infty & -\infty \\
    \end{bmatrix} $
\end{itemize}

运行\texttt{iteration\_bound.m}可得输出:
\begin{itemize}
    \item \texttt{iteration\_bound = 4}
\end{itemize}

% subsection problem_3 (end)

\subsection{Problem 5} % (fold)
\label{sub:problem_5}

运行\texttt{p5\_graph.m}可得输入:
\begin{itemize}
    \item $d=7$
    \item $L^{(1)} = \begin{bmatrix}
        16 & 16 & -\infty & -\infty & 16 & -\infty & -\infty \\
        -\infty & 4 & -\infty & -\infty & -\infty & -\infty & -\infty \\
        15 & 15 & -\infty & -\infty & 15 & -\infty & -\infty \\
        14 & 14 & -\infty & -\infty & 14 & -\infty & -\infty \\
        -\infty & -\infty & -\infty & -\infty & 4 & -\infty & -\infty \\
        14 & 14 & -\infty & -\infty & 14 & -\infty & -\infty \\
        14 & 14 & -\infty & -\infty & 14 & -\infty & -\infty \\
    \end{bmatrix} $

    (按图中自左到右顺序依次将D节点编号为D\textsubscript{1}到D\textsubscript{7})
\end{itemize}

运行\texttt{iteration\_bound.m}可得输出:
\begin{itemize}
    \item \texttt{iteration\_bound = 16}
\end{itemize}

% subsection problem_5 (end)

% section 问题求解 (end)
\newpage
\section*{A. 附录} % (fold)
\label{sec:附录}

\small 所有源码已上传至 \texttt{https://github.com/Nuullll/iteration-bound}

\subsection*{\texttt{iteration\_bound.m}} % (fold)

\lstinputlisting[language=Matlab]{../src/iteration_bound.m}

\subsection*{\texttt{p5\_graph.m}} % (fold)

\lstinputlisting[language=Matlab]{../src/p5_graph.m}

\subsection*{\texttt{findLongestPath.m}} % (fold)

\lstinputlisting[language=Matlab]{../src/findLongestPath.m}

\subsection*{\texttt{LPM.m}} % (fold)

\lstinputlisting[language=Matlab]{../src/LPM.m}

\subsection*{\texttt{getIterationBound.m}} % (fold)

\lstinputlisting[language=Matlab]{../src/getIterationBound.m}

% section 附录 (end)

\end{document}